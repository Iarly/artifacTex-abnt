%
% Documento: Introdução
%

\chapter{Introdução}\label{chap:introducao}

Esta é a primeira página\index{alfa}{pagina@página} na qual a numeração é visível (canto superior direito).
Entretanto, a numeração das páginas (neste modelo de documento\index{alfa}{documento}) começa na Folha de Rosto, que é a página\index{alfa}{pagina@página} 1. Toda a numeração de página\index{alfa}{pagina@página} é automática.

Cada capítulo deve conter uma pequena introdução (tipicamente, um ou dois parágrafos) que deve deixar claro o objetivo e o que será discutido no capítulo, bem como a organização do capítulo. Veja o exemplo abaixo.

Neste capítulo apresenta-se o desenvolvimento\index{alfa}{desenvolvimento} da \ldots, no que se refere à \ldots, estudos e autores importantes nesta trajetória. Para contextualizar a \ldots\mbox{} no âmbito dos estudos sobre \ldots, faz-se, na seção 1.1, uma introdução aos princípios-chave \ldots\mbox{ }Na seção 1.2, discute-se os aspectos relacionados \ldots, sob o ponto de vista da \ldots\mbox{ }blá blá \ldots\mbox{ }Ao final, seção 1.N, apresenta-se alguns comentários e uma síntese final do capítulo.

A inclusão de reticências (\ldots) no texto deverá ser feita através de um comando especial denominado \verb|\ldots|. Assim esse comando deverá ser utilizado ao invés da digitação de três pontos.

Para melhor entendimento do uso do estilo de formatação, aconselha-se que o potencial usuário analise os comandos existentes no arquivo {\ttfamily main.tex} e os resultados obtidos no arquivo {\ttfamily main.pdf} depois do processamento pelo software LATEX + BIBTEX \cite{LaTeX2009,BibTeX2009}. Recomenda-se a consulta ao material de referência do software para a sua correta utilização \cite{Lamport1986,Buerger1989,Kopka2003,Mittelbach2004}.

\section{Motivação}
\label{sec:motivacao}

Uma das principais vantagens do uso do estilo de formatação para LATEX é a formatação \textit{automática} dos elementos que compõem um documento acadêmico, tais como capa, folha de rosto, dedicatória, agradecimentos, epígrafe, resumo, abstract, listas de figuras, tabelas, siglas e símbolos, sumário, capítulos, referências, etc. Outras grandes vantagens do uso do LATEX para formatação de documentos acadêmicos dizem respeito à facilidade de gerenciamento de referências cruzadas e bibliográficas, além da formatação -- inclusive de equações matemáticas -- correta e esteticamente perfeita.

\section{Caracterização do Problema}
\label{sec:caracProblema}

Inserir seu texto aqui...

\section{Objetivos}
\label{sec:objetivos}

\subsection{Objetivo Geral}
\label{subsec:objetivoGeral}

Prover um modelo de formatação LATEX que atenda às normas da instituição atual e às normas brasileiras.

\subsection{Objetivos Específicos}
\label{subsec:objetivosEspecificos}

\begin{itemize}
	\item Obter documentos acadêmicos automaticamente formatados com correção e perfeição estética.
	\item Desonerar autores da tediosa tarefa de formatar documentos acadêmicos, permitindo sua concentração no conteúdo do mesmo.
	\item Desonerar orientadores e examinadores da tediosa tarefa de conferir a formatação de documentos acadêmicos, permitindo sua concentração no conteúdo do mesmo.
\end{itemize}


\section{Organização do Documento}
\label{sec:organizacaoDocumento}

Inserir seu texto aqui...

\section{Justificativa}
\label{sec:justificativa}

Inserir seu texto aqui...
