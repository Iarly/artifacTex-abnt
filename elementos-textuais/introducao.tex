%
% Documento: Introdução
%

\chapter{Introdução}\label{chap:introducao}

O presente documento é um exemplo de uso do estilo de formatação LATEX elaborado para atender às Normas para Elaboração de Trabalhos Acadêmicos. O estilo de formatação {\ttfamily abnt-cefetmg.cls} tem por base o pacote ABNTEX -- cuja leitura da documentação \cite{abnTeX2009} é fortemente sugerida.

Para melhor entendimento do uso do estilo de formatação, aconselha-se que o potencial usuário analise os comandos existentes no arquivo {\ttfamily main.tex} e os resultados obtidos no arquivo {\ttfamily main.pdf} depois do processamento pelo software LATEX + BIBTEX \cite{LaTeX2009,BibTeX2009}. Recomenda-se a consulta ao material de referência do software para a sua correta utilização \cite{Lamport1986,Buerger1989,Kopka2003,Mittelbach2004}.

\section{Motivação}
\label{sec:motivacao}

Uma das principais vantagens do uso do estilo de formatação para LATEX é a formatação \textit{automática} dos elementos que compõem um documento acadêmico, tais como capa, folha de rosto, dedicatória, agradecimentos, epígrafe, resumo, abstract, listas de figuras, tabelas, siglas e símbolos, sumário, capítulos, referências, etc. Outras grandes vantagens do uso do LATEX para formatação de documentos acadêmicos dizem respeito à facilidade de gerenciamento de referências cruzadas e bibliográficas, além da formatação -- inclusive de equações  matemáticas -- correta e esteticamente perfeita.

\section{Caracterização do Problema}
\label{sec:caracProblema}

Inserir seu texto aqui...

\section{Objetivos}
\label{sec:objetivos}

\subsection{Objetivo Geral}
\label{subsec:objetivoGeral}

Prover um modelo de formatação LATEX que atenda às normas da instituição atual e às normas brasileiras.

\subsection{Objetivos Específicos}
\label{subsec:objetivosEspecificos}

\begin{itemize}
	\item Obter documentos acadêmicos automaticamente formatados com correção e perfeição estética.
	\item Desonerar autores da tediosa tarefa de formatar documentos acadêmicos, permitindo sua concentração no conteúdo do mesmo.
	\item Desonerar orientadores e examinadores da tediosa tarefa de conferir a formatação de documentos acadêmicos, permitindo sua concentração no conteúdo do mesmo.
\end{itemize}


\section{Organização do Documento}
\label{sec:organizacaoDocumento}

Inserir seu texto aqui...

\section{Justificativa}
\label{sec:justificativa}

Inserir seu texto aqui...
