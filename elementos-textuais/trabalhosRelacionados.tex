%
% Documento: Trabalhos Relacionados
%

\chapter{Trabalhos Relacionados}

Este capítulo inclui muitas citações bibliográficas. Os principais
itens de bibliografia citados são livros, artigos em conferências,
artigos em {\textit journals} e páginas Web. A bibliografia deve seguir o
padrão ABNT\footnote{Este não é o endereço oficial da
ABNT pois as Normas Técnicas oficiais são pagas e não estão disponíveis na Web.}.

A bibliografia é feita no padrão {\ttfamily bibtex}.  As referências são
colocadas em um arquivo separado. Os elementos de
cada item bibliográfico que devem constar na bibliografia são
apresentados a seguir.

Para livros, o formato da bibliografia no arquivo fonte é o seguinte:

\begin{verbatim}
    @Book{linked,
      author =       {A. L. Barabasi},
      title =        "{Linked: The New Science of Networks}",
      publisher =    {Perseus Publishing},
      year =         {2002},
    }
\end{verbatim}

A citação deste livro se faz da seguinte forma \verb#\cite{linked}# e o resultado fica assim \cite{linked}.
Para os artigos em {\textit journals}, veja por exemplo \cite{acmsurveys},
descrito da seguinte forma no arquivo {\ttfamily .bib}:

\begin{verbatim}
   @article{acmsurveys,
     author = {Deepayan Chakrabarti and Christos Faloutsos},
     title = {Graph mining: Laws, generators, and algorithms},
     journal = {ACM Computing Surveys},
     volume = {38},
     number = {1},
     year = {2006},
     pages = {2-59},
     publisher = {ACM},
     address = {New York, NY, USA},
   }
\end{verbatim}

O artigo \cite{3faloutsos} foi publicado em conferência.  Embora
às vezes seja difícil distinguir um artigo publicado em {\textit
  journal} de um artigo publicado em conferência, esta distinção é
fundamental.  Em caso de dúvida, procure ajuda de seu orientador.

Veja também duas citações juntas \cite{rp99,mar00}  e como citar
endereços Web \cite{irl:06}.
O trabalho realizado para editar as citações no formato correto é
compensado por uma bibliografia impecável.
